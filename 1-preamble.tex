%!TEX root=paper.tex

\begin{abstract}
  % UPDATED---\today. 

 % This paper describes 
 % the design, implementation,
 % usage analysis, and evaluation of 
 We present 
 % a ``personalized language textbook'' -- 
 a system designed to enable learners of a foreign language to 
   read materials that are personally interesting
     to them from the web and 
  practice vocabulary with interactive exercises based on their past readings. 
  % The system does this 
  % by allowing them to 
  % read news and blogs, sourced
  % from the Internet, in 
  % An interactive reader provides seamless translations 
  % for the unknown words and based on past interactions
  % with texts 
  % while at the same
  % time 
  %     saving them in order to 
  %        be able to monitor the current state of the knowledge
  %         of 
  % the learner is presented with personalized exercises
        % which are derived from past readings.
  We report on the results of deploying the
  system for one month with three classes of Dutch highschool students 
  learning French. 
  The students and their teacher were positive about the system
  and in particular about the personalization aspects
  that the system enables. 
  % The teacher has decided to redeploy the system
  % for the next academic year.

\end{abstract}

\category{H.5.2.}{Information Interfaces and Presentation
  (e.g. HCI)}{User Interfaces}{}{}

\keywords{\plainkeywords}
