%!TEX root=paper.tex

\section{Introduction}
% \ml{still too ``brainstormy''... }
% language learning - which in itself is a very broad issue with the potential to impact the lives of a very large number of people: 
The British Council estimates that by 2020 there will be 2 billion people learning English as a foreign language. When learning a new language, reading is considered to act as a microcosm of all the other skills \cite{mccarthy1999-microcosm}. Reading materials that are interesting to the learner will increase their desire to read and increase the time spent reading.  

% However, to enjoy the benefits of free reading, the learner must already be sufficiently fluent in the target language: even knowing 95\% of the vocabulary in a text, a learner still has to look up in average a word on every line. \cite{Hirsh92-vocab-size} This means that not any text is good for reading. Randomly choosing a text might actually end up being frustrating because the learner has to look up a lot of words in the dictionary.

Before fluency, learners use language textbooks as reading material. Such textbooks are an artefact of the last century which still proves to be useful today. They are designed by experts who make sure that the texts that the readers are reading are simple enough for the desired language level of a broad audience. One of their advantages is that they have exercises which are based on the texts. Over the years, the textbooks have become more colorful, they come with complementary audio or video lessons, but their main limitation is unchanged: 
% static nature has not changed.
% essence they still are a collection of texts with associated exercises.
% We believe that one of the main limitations of the textbook approach stems from the fact that textbooks 
by being designed for a average generic learner they are not exciting for any individual learner. A student passionate about sports, might read with pleasure ten articles on sports rather than one about ``Maria, who is a babysitter in Spain''. 
% As the US Air Force learned the hard way sixty years ago, ``there is no average pilot'': when cockpits, jumpsuits, and instructions were designed for the {\em average pilot}, the actual pilots had a hard time maneuvering the planes; performance improved only when the cockpit was designed in such a way as to be adjustable to the individual. 
% As a result, the motivation of the individual student for using textbooks is low. 

% The fact that textbooks do not work that well is illustrated also by the fact that some of the teachers of foreign languages whom we have spoken to, do not even use a textbook anymore, but instead find articles that they deem interesting online and share them with their students. Note that this could be a step in the right direction, since the teacher has a better understanding of the interests of the class. However, in the end, the student is still not reading what they are passionate about, but rather, what the teacher deems relevant. 

Given the vast amounts of multi-language content available on the Internet, it is likely that every student can find materials that are personally interesting for them from the plethora of blogs, news articles, eBooks that exist and appear daily in all the major languages. 
% 
% 
% The texts that one can read are limited to generic topics, which must appeal to the entire audience. Unfortunately a text that is good enough for everybody is likely not exciting for anybody. This limits the amount of reading that learners will do, and thus, limits their learning.
% 
% If there was a way to allow every reader to find and study materials that are interesting for them the motivation of the students would be highly increased. People could even 
% 
% 
% 
% 
However, three problems prevent materials on the Internet from replacing traditional textbooks for the intermediate student:

\begin{enumerate}
	

  \item Comprehension support is not always appropriate. The optimal comprehension support infrastructure should ideally work {\em ``without requiring even a single click from the user''} \cite{Proszeky02-Comprehension}. 
  Complicated interactions (e.g. external dictionaries) that interrupt the flow of reduce the enjoyment of reading.
  
  \item Reading tools are not integrated with exercise platforms to ease a learner's vocabulary practice. Instead learners usually have to manually add words to an external vocabulary practice platform. 

  \item Online reading platforms lack information about text difficulty. A learner that picks an article randomly might choose an article that is too difficult and eventually give up.\footnote{E.g., the german articles in the  daily {\em Neue Z\"uricher Zeitung} have a very high degree of difficulty variability: some could be read by advanced beginners and many would be too difficult for advanced intermediates.}

  
\end{enumerate}


There are many other domains where the averages are being replaced with individualized attention: medicine\footnote{The nascent discipline of {\em personalized medicine} suggests that analysis of the genetic makeup of an individual may guide health care decisions far more precisely than big group studies do}, computer security,  web design \cite{Reinecke13-CulturalAdaptation}, mathematical education \cite{Polozov15-AdaptableMath}. 

% In this paper we are focusing our attention on a subset of education - 


In order to address these three problems, we propose an infrastructure which allows learners to: 

	\begin{description}
	
		\item [Exert agency over the materials they study] by selecting written content that are interesting for them
	
		\item [Access conveniently translations for unknown words] in those cases when they are encountered, as it is unlikely that these words can be completely avoided.

		\item [Practice using personalized \& contextual exercises] that are generated automatically based on their past readings.

	\end{description}

% Although it is not the focus of this paper, the infrastructure should also Estimate the difficulty of a text in order to allow the learners to avoid materials that are too difficult. \ml{maybe don't even mention this... we're setting ourselves up for too much}

% In the remainder of this paper we describe the design of such a system (Section \ref{sec:system}) and we present our results from deploying the system for one month with a group of \students Dutch high school French learning students (Sections \ref{sec:demographics} -- \ref{sec:perception}). We then talk about the limitations of this study (Section \ref{sec:limitations}) and then we list some of the challenges that we think our infrastructure and similar ones must face in order to increase the chance of their acceptance in practice (Section \ref{sec:challenges}).

