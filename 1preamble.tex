%!TEX root=paper.tex

\begin{abstract}
  UPDATED---\today. 

 This paper describes the design, implementation,
 usage analysis, and evaluation of a ``personalized
 language textbook'' -- 
 a system that supports learners of a foreign language 
  in reading materials that
  are personally interesting by allowing them to 
  read in an interactive reader news and blogs sourced
  from the Internet.
  The system provides translations for the unknown 
  words at the touch of the screen while at the same
  time 
      saving them in order to 
         be able to monitor the current state of the knowledge
          of the learner 
      and to later generate personalized exercises
        which are derived from past readings.

  This paper reports on the results of deploying the
  system for one month with three classes of Dutch highschool students 
  learning French. 
  The students and their teacher were overall very positive about the system, 
  and in particular about the study personalization aspects
  that it enables. 
  % The teacher has decided to redeploy the system
  % for the next academic year.

\end{abstract}

\category{H.5.m.}{Information Interfaces and Presentation
  (e.g. HCI)}{Miscellaneous} \category{See
  \url{http://acm.org/about/class/1998/} for the full list of ACM
  classifiers. This section is required.}{}{}

\keywords{\plainkeywords}
