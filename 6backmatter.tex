%!TEX root=paper.tex


\newpage
\section{Challenges}

In this section we explore some of the challenges that we percieve need to be addressed by our system and similar personal textbook systems.

\subsection{The lack of editorial work}
The advantage of a textbook is the fact that the quality control is guaranteed. How are we going to ensure the quality of the texts, and the quality of the generated exercises? 

\subsection{The selection of vocabulary to study}

How do we automatically verify the ``learnability '' of an example in the context? It is a great responsibility automatically selecting a word to study. The situation where a user accepted a mistaken translation, and then the system ``teaches'' the learner that word would be disastruous. Currently we have a set of filters that try to avoid this, moreover, we also offer the learner the possibility of providing feedback in case he is not confident in a given translation. In the future, we consider using crowdsourcing to decrease even more the probability of wrong translations.
\ml{alternate perspecitve: Do the learners choose the right translation? }


\subsection{Evaluating the quality of an example}

It is indeed desirable to find good examples of practice exercises from past readings. Sometimes, the context in which the learner looks up a word is too long and sometimes it is too short. How estimate the quality of an exercise? One measure that we are considering is: ensuring that all the words in the context are simpler than the looked up word. 

% For beginners, this is still not an option. So we can only do this for students who are already quite advanced. \todo{We should see whether there's a difference between the ones that were A2 vs. B1}

\subsection{Difficulty Ranking}
The way we did difficulty ranking was sub-optimal. Most of the difficulties were very close to each other in value, between 1 and 4... This is too abstract, and hard for a reader to understand what it means. A more advanced strategy is needed. One approach would be to estimate the number of words that will need translation. Antoher would be to also show the number of words that are being learned at the moment, such that the learner, can choose a text that also gives them the chance to study. 

\subsection{The teacher perspective}
The system we presented here has a (limited) teacher dashboard for those users who have a teacher. The dashboard currently shows a chronological log of the words that the student has looked up in the context. However, the system could present more advanced analytics that could enhance the teacher's understanding of the class. This is something that is a clear opportunity when moving to a digital textbook. 

\subsection{Integrating the system in an classroom workflow}
The system was initially designed for self study. However, when invited to test it in a formal classroom we were happy to oblige. We plan to work more with teachers to better understand how to combine the individuality of the system with the shared experience of the learners in a classroom. Indeed, new workflows and classroom activities must be discovered.

\subsection{A longitudinal study}
The study with students that we report here has been done with \stcnt students for less than a month. Based on the experience, the teacher of the class decided that they want to introduce the system in the next academic year with a slightly larger group of students -- about one hundred and thirty for the entire academic year. This will give us a wealth of data that we plan to record and analyze the usage of the system and hope to learn even more about the possibilities and limitations of such a system. 

\subsection{Registering for ``topics'' instead of ``sources''}
See whether allowing people to register to topics rather than sources (i.e. newspapers) would make more sense for them. 




\section{Availability}

\subsection{Deployed System}
The system described in this paper is deployed online and available at \url{https://zeeguu.unibe.ch/signup}. If the readers of this article want to test it they can use the {\em CHI2018} invite code while following the  ``Become a Betatester'' link from the homepage.

\subsection{Data}
The anonymized data representing the user interactions of the \students that used the system for one month is available as a MySQL database dump on GitHub at the following link: \url{https://github.com/mircealungu/chi17}.


