%!TEX root=paper.tex

\newpage
\section{Testing With High School Students}
\label{sec:demographics}

We tested our system with \stcnt students from a public highschool in Groningen, The Netherlands. They represent three classes that have the same French teacher and are bilingual in Dutch and English. At the beginning of June 2017, we visited the school, and during one hour in each of the three classes we introduced the tools and their usage.

The system was used officially in class from the second week of June until the end of the month (coinciding with the end of the semester). With few exceptions the students created an account and started using the system the latest on June 9th. 

The teacher asked the students to use the infrastructure as much as they liked while preparing at home and to write reports on their activity. For every half an hour of usage, the students had to write a brief report on how they spent their time and submit it to the teacher. The teacher could then decide to selectively test them on the basis of their reports.

We deployed the system with the translations from French to English instead of Dutch since, based on our experience translation APIs are of higher quality when one of the languages is English and because the students and their teacher were comfortable with the idea.\footnote{We made it clear to the students that they can ask us, and we will modify their personal account in such a way as to receive translations in Dutch. None of the students requested this.}

We also invited the students to send us feedback at any time if they encounter problems or if they have ideas for improvement. Several of them did email. Towards the end of the month, we also deployed several in-app focused pop-up questions using a customer opinion elicitation service called HotJar. After the semester was over we sent out a follow-up questionnaire.

\subsection{Demographics}

Before creating accounts on our platform, the participants were directed to a survey form which asked them to provide personal information about their current level of knowledge, learning strategies, and interests. A handful of the participants did not fill the survey before using the system.

The participants that filled our survey were 54 female and 15 male with ages below 18 representing three different classes. Based on their own self characterization, 53 students are level B1 (i.e. can understand the main points of clear standard speech, can narrate an event, an experience or a dream) and 16 are level A2 (i.e. using simple words, can describe their surroundings and communicate immediate needs). 


When asked whether they have favorite topics they would like to read about, half of the students mentioned such topics while the other half did not answer the question. From the topics that they mentioned as possible interests some of the more popular were: sports, music, travel, lifestyle, fashion, movies, and somebody mentioned as interest {\em ``no politics''}.

We seeded the system with a variety of French news and blogs that cover these aspects: 1Jour1Actu, L'Equipe, La Blogoteque, Le Figaro, Le Monde. 











% To talk to Nienke about the other types of analysis we can do on this data