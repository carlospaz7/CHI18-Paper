%!TEX root=paper.tex

\newcommand{\q}[1]{($Q_#1$)}
\newcommand{\qq}[2]{($Q_#1,Q_{#2}$)}

\newpage
\section{The Perception of the Teacher}
% \section{What Is the Perception of the Teacher?}
After the deployment at the school was finished we conducted a semi-structured interview with the language teacher of the three classes to gain insight into his perception of the benefits and limitations of the system. 
% 
The teacher, who self-describes as {\em an experienced teacher and language scientist as well}, argues that that: 

\begin{itemize}

	\item Such a system is critical for language education in schools, since the possibility of choosing their topics of interest is motivating for the students, and motivation is critical \q{3}\footnote{The $Q_n$ annotations refer to the questions in the full text of the interview which is available online at: \url{https://github.com/zeeguu-ecosystem/CHI18-Paper/blob/master/data/teacher-interview.txt}}. 

	\item The system should be used for students who had already two or three years of foreign language experience \q{5}. 

	\item There is no danger that every student will develop his little individual vocabulary bubble. Once the students have a solid basic vocabulary, it is perfectly acceptable that they study the words which interest them. \q{5}

	\item The used sources were maybe too general. One source which was very focused on sport news was found very interesting by several students, mostly boys. More highly specific sources for other topics could be good \q{7}.

	% \item usually I made them learn vocabulary by heart for a test but this has proven to be rather useless because it only improves their short-term memory.  the good thing about your system is that they encounter a lot of words a lot of times and also you have spaced repetition and that is very good for learning vocabulary


	\item It is ``more than acceptable'' that the translations are not perfect and every now and then a student must look up an alternative translation. This might help students become more actively engaged with the texts \q{9}. 
	% If 98% of the translations are correct they only click on the world and they continue reading but when they look at the word and have to think about whether is the right translation or not, they are more active. You

	% 
	% students have been studying French for 3 years the most frequent vocabulary they already have. I am an experienced teacher and language scientist as well; I know that when you have this base vocabulary and you go beyond it most of the world that just in the text are still the same, a small proportion of the words the people use. If you look at the French language, there are 25’000 words but when you're reading an article, you only encounter very few of those 25’000 words. When students read different texts, I think in the end 75% of the words that they will learn will be the same. 

	\item The most important missing feature of the system is the possibility of comprehensively verifying that the students put quality and time in using the system at home \qq{2}{11}.

\end{itemize}


The teacher decided to extend the use of the system during the academic year 2017--2018 with a larger group of students. 
% However, not all the new classes are bilingual, we will have to explore the best approach for situations in which the L1 is not English but Dutch.
