%!TEX root=paper.tex

\newpage
\section{Introduction}

It is known that, when learning a new language, free reading is one of the best ways of improving ones vocabulary. Reading something that's interesting to the learner will increase their desire to read \ml{a prof in california was arguing this I think}. And reading is important since it is a microcosm of all the other skills. \cite{mccarthy1999-microcosm} 

However, to enjoy the benefits of free reading the learner must already be quite fluent in the target language: even knowing 95\% of the vocabulary in a text means that a learner has to look up a word on every line of text. \cite{Hirsh92-vocab-size}

Before fluency, most of the times, learners use language textbooks as reading material. Textbooks are an artefact of the last century which still proves to be useful today. They are designed by experts who make sure that the texts that the readers are reading are simple enough for the desired level and interesting enough for a broad audience. One of their advantages is that they have exercises which are based on the texts that the readers read. Over the years, the textbooks have become more colorful, they even come with complementary audio lessons, but in their essence they still are a collection of texts with associated exercises.

We believe that one of the main limitations of the textbook approach stems from the fact that they are designed for the average learner and they can not adapt to the individual. As the US Air Force learned the hard way sixty years ago, ``there is no average pilot'': when cockpits, jumpsuits, and instruction were designed for the {\em average pilot}, the actual pilots had a hard time maneuvering the planes; performance improved only when the cockpit was designed in such a way as to adjustable to the individual. 

There are many other domains where the averages are being replaced with individualised attention: medicine\footnote{The nascent discipline of ``personalized medicine'' suggests that analysis of the genetic makeup of an individual may guide health care decisions far more precisely than big group studies do}, computer security \ml{cite that paper from the lab notebook maybe}, and education. In this paper we are limiting our attention to a subset of education, language learning, which in itself is a very broad issue with the potential to impact the lives of a very large number of people: the British Council estimates that by 2020 there will be 2 billion people learning only English as a foreign language. 

We can see a few ways in which the design of the language education system for the average is hurting the individual in the domain of reading foreign language materials: 

\begin{itemize}
	\item Average Interest. The texts that one can read are limited to generic topics, which must appeal to the entire audience. Unfortunately a text that is good enough for everybody is likely not exciting for anybody. This limits the amount of reading that learners will do, and thus, limits their learning.
	\item Average Knowledge.  Every learner has a different prior knowlege and a different mix of already known languages. 
	\item Average Pace. ... 
\end{itemize}

% If there was a way to allow every reader to find and study materials that are interesting for them the motivation of the students would be highly increased. People could even 

The fact that textbooks do not work that well is illustrated also by the fact that some of the teachers of foreign languages that we have spoken to, do not even use a textbook anymore for their students, but instead find articles online and share them with their students \ml{to ask Wim for details.} This clearly does not solve the problems that we listed before, but at least, it solves the problem for the teacher, who can now find a text that is interesting for him, and he hopes willbe interesting for the students.

However, the limitation of the textbooks do not apply to the vast amounts of information available on the Internet in all the possible languages that somebody would want to learn. Blogs, News, eBooks exist for all the major languages. A student passionate about sports, might read with pleasure 10 articles of sport rather than one about ``Maria who is a babysitter in Spain''. However, there are three problems that prevent readers to read materials on the Internet:

\begin{itemize}
	
  \item The materials might be too difficult for them. The articles in the  daily {\em Neue Z\"uricher Zeitung} have a very high degree of variability in their textual difficulty. A learner that picks an article randomly might choose an unpleasantly difficulty article.

  \item The existing reading tools might not be appropriate. Given that people are less likely to buy a product if they have to click multiple times to have it delivered to their home\footnote{Ergo the Amazon patent for 1-click buy, and the fact that Apple licensed it from them} it is very likely that people will shy away from doing more than a click to obtain a translation for an unknown word.

  \item The texts available on the Internet do not come with exercises that would help a learner retain newly learned words, and improve their own vocabulary. 
  
\end{itemize}

We believe that education could be made much more personalized, and more pleasurable, if there was a way to allow the learners to:
\begin{itemize}
\item Express their interest in materials that are interesting for them.
\item Be presented with texts which are at the right difficulty level.
\item Have access to a very convenient way of translating unknown words in those cases when they are incident.
\end{itemize}

On top of that, we could also design personalized exercises that would be generated automatically based on the past reading of the students,

Finally, textbooks are also designed with exercises which help settle the words in such a way that the words get repeated.

Moreover, textbooks usually come with exercises at the end of a text. However, if one allows the readers to read whatever they like: how are they going to do exercises? Can we generate exercises that are tailored based on the past experience of students free reading?

\paragraph{Contributions}
The contributions of this paper are: 
\begin{itemize}
	\item A minimal architecture that enables such a system
	\item Results from deploying the system with seventy students for about one month
\end{itemize}
