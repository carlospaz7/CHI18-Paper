%!TEX root=paper.tex

\section{Related Work}
\ml{incomplete. must finalize}


\subsection{Augmenting Foreign Texts}
The idea of augmenting texts with translations has been proposed before in various forms. One of the first occasions was in the work of Nerbonne \cite{Nerb99-Assistant} who propose Glosser -- system, which would provide dictionary information about a given word, including translation, part of speech, declinations. etc. In their work they focus on individual words; however, it is often the case that while reading a foreign text, a reader might want to translate a sequence of words. In a study with 22 people they observed users using the system for twenty minutes \cite{Dokter98-UserStudy}.

Azab et al. \cite{Azab13-nlp} have proposed a system entitled SmartReader which provides interactive annotations of English words for the advanced foreign students who learn English. Popups are displayed above the selected word with information about it. The study introduces and describes the system it does not report anything about the way the system is being used by users.

\subsection{Augmenting Native Texts}
In a different direction than the previous work, Trusty and Truong augment the web in a learners native language with translations of a fixed set of words in the language that they are learning\cite{Trus11web}. They show that in a two month deployment, 18 participants, learned in average 50 new words.

\subsection{Vocabulary Practice Exercises}
The number of systems that can provide vocabulary exercises to the learners is very large with several very popular commercial systems such as Babbel, DuoLingo, RosettaStone, Memrise, etc. Babbel offers 2000 to 3000 words per language, with is comparable to most Duolingo courses. Rosetta Stone claims that you can reach up to C1 with their advanced course. The system we present here can be used by any learner, no matter how advanced. Moreover, the example sentences that one practices with are pre-defined, not personalized.

One of the advantages of the predefined example systems is that they can teach also grammar. DuoLingo is designed in such a way that its exercises are designed to teach also grammar. Duo doesn't teach you grammar; it only drills you on it. 

Spaced repetition systems like Anki are good at drilling vocabulary.

\subsection{Personalization}

In web design Reinecke et al. propose culturally adaptive interfaces which able to adapt their look and feel to suit visual preferences of a given population \cite{Reinecke13-CulturalAdaptation}. 

In mathematical education, Polozov et al. propose a technique for automatic generation of personalized word problems\cite{Polozov15-AdaptableMath}.









