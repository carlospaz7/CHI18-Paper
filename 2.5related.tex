%!TEX root=paper.tex
\newpage
\section{Related Work}

The domain of computer assisted language learning has a rich history of applied research that aims to improve the effectiveness and efficiency of language learning through helping both teachers and students \cite{levy2013call}. In this discussion we focus on several aspects that we combine in this work, which we argue, have not been combined together before.


\subsection{Using the Web as A Source of Language Materials}
% Web Mining

Multiple authors have observed before that the World Wide Web represents an enormous language database at the fingertips of the students \cite{Fried08-Learner,Hira07-WebCorpora}. 
Thus, the idea of augmenting texts with translations has been proposed before in various forms: 

\begin{itemize}

	\item One of the first occasions was in the work of Nerbonne \cite{Nerb99-Assistant} who proposed Glosser -- a system which would provide dictionary information about a given word including translation, part of speech, declinations, etc. 
	In a follow up study with 22 people they observed users using the system for twenty minutes \cite{Dokter98-UserStudy}. 
	In their work, they focus on individual words. 
	In our work we observed a larger number of learners for a longer period of time.
	% TODO: verify this!
	In their study the words were previously annotated; our tools allow learners to translate sequences of words and not only individual words. 

	\item Azab et al. \cite{Azab13-nlp} proposed a system entitled SmartReader which provides interactive annotations of English words for the advanced foreign students who learn English. 
	Pop-ups are displayed above the selected word with information about it. 
	The study introduces and describes the system, however it does not report anything about the way the system is used.

	\item Wible et al. \cite{Wible01-Exposure} introduce SRP -- a stand-alone tool that provides teachers and students with search capabilities for supplementary readings online.
	The tool exploits text retrieval techniques and is based on the hypothesis that there is a parallel between text similarity measurement on the one hand and the pedagogical task of providing supplementary readings which offer repeated exposure to new vocabulary.
	% The aim of SRP is to take a target vocabulary item as input and provide as output a set of texts from the corpus that contain tokens of the target vocabulary which resemble the original semantically and of course match it in part of speech.
	However, the system is presented without any evaluation with users.

	\item Streiter et al. \cite{Streit05-Browsers} argued for a a system that would support browsing the Internet and a local document repository by dynamically annotating HTML and PDF documents with open dictionaries resources. A word is annotated with translations and pictures. It argues for the creation of personal word lists and exdrcises. But there is no report of the system being evaluated with users. 

	\item Trusty and Truong augmented the web in a learners native language with translations of a fixed set of words in the language that they are learning\cite{Trus11web}. They show that in a two month deployment, 18 participants, learned in average 50 new words.

\end{itemize}


\subsection{Interactive Texts}

Augmenting foreign texts with annotations in the form of pop-ups, and overlays, has been found to be beneficial to several aspects of language learning \cite{DeRidder02-Links} and improvements in reading comprehension \cite{Sanko06-Effects}.


\begin{itemize}
	\item DeRidder \cite{DeRidder02-Links} studied the behavior of students reading with hyperlinks. The results indicate that when reading a text with highlighted hyperlinks, readers are significantly more willing to consult the gloss. Sanko \cite{Sanko06-Effects} showed that hypertextual input enhancement favourably affects vocabulary learning.


	\item Yang showed that multimedia gloss groups are useful in retention: they noticed that users using multimedia glosses retained and recognized significantly more of the target words than the control group \cite{Yang09-Glosses}.


	\item D\'iaz \cite{Diaz15-Augmented} did a study of how the users augment web browsers with extensions in order to ``personalize on demand'' their browsing experience. Based on millions of web users they saw that Google Translate was the 16th most used browser extension.


	\item Horv\'ath conducted a small supervised experiment to evaluate effect of text
	augmentation of reading speed. The results show that augmented webpage slows
	reading speed down on average by approximately 7\%. \cite{Horva13-Enriching} 


\end{itemize}



\subsection{Vocabulary Practice}

The number of systems that can provide vocabulary exercises to the learners is very large with several very popular commercial systems such as Babbel, DuoLingo, RosettaStone, Memrise, etc. being mainly focused on beginners and early intermediates. 

Most of the systems personalize the exercises based on the learning progress of the users but not the content. Contentwise, these systems have predefined content, or no content at all (e.g. Anki, Memrise) and the user is supposed to manually punch in the content. 

Dasgupta argues that in the context of interactive books, self-contained exercises to be included \cite{Dasgupta10-Play}. Also studies show that learning a word in context is more effective \cite{nagy95-context}.



% Babbel and DuoLingo offer 2000 to 3000 words per language. 
% Rosetta Stone claims that one can reach past B2 with their advanced course. 
% Almost all these previous systems are targeted at the beginners and intermediates.
% The system we present here can be used by any learner, no matter how advanced. 


% The main limitation of all these afore-mentioned systems is that the example sentences that one practices with are pre-defined. 
% We aim for personalization.
% One of the advantages on the other hand, is that predefined example systems is that they can also teach grammar, which is something we do not investigate at the moment.

Recently, creative applications of vocabulary exercises:
 % have been proposed in the context of micro-learning systems: 

\begin{itemize}

	\item Dearman and Truong propose a 'live wallpaper' interface that is always visible to the user when he is using his phone \cite{Dear12-ImplicitAcquisition}. They also present words in context. 

	\item WaitChatter takes advantage of micro-learning opportunities and presents vocabulary exercises while the user awaits IM responses \cite{Cai15-wait}.

\end{itemize}




\subsection{Personalization}

The idea of a Personal eLearning Environment has been proposed by Attwell in 2007 \cite{Atwell07-personal} who assumes that it will take place in different contexts and situations and will not be provided by a single learning provider.

In web design Reinecke et al. propose culturally adaptive interfaces which are able to adapt their look and feel to suit visual preferences of a given population \cite{Reinecke13-CulturalAdaptation}. 

In mathematical education, Polozov et al. propose a technique for automatic generation of personalized word problems\cite{Polozov15-AdaptableMath}.









