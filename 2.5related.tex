%!TEX root=paper.tex

\newpage
\section{Related Work}
% \ml{incomplete. must finalize}


\subsection{Augmenting Foreign Texts}
The idea of augmenting texts with translations has been proposed before in various forms. One of the first occasions was in the work of Nerbonne \cite{Nerb99-Assistant} who proposed Glosser -- system, which would provide dictionary information about a given word, including translation, part of speech, declinations, etc. In a follow up study with 22 people they observed users using the system for twenty minutes \cite{Dokter98-UserStudy}. In their work, they focus on individual words. In our work we observed a larger number of learners for a longer period of time, and our tools allow them to translate sequences of words and not only individual words. 
%maybe some conclusion here

Azab et al. \cite{Azab13-nlp} proposed a system entitled SmartReader which provides interactive annotations of English words for the advanced foreign students who learn English. Pop-ups are displayed above the selected word with information about it. The study introduces and describes the system, however it does not report anything about the way the system is used.

\subsection{Interactive Texts}
In a different direction than the previous work, Trusty and Truong augmented the web in a learners native language with translations of a fixed set of words in the language that they are learning\cite{Trus11web}. They show that in a two month deployment, 18 participants, learned in average 50 new words.

Dasgupta argues that in the context of interactive books, self-contained exercises to be included. \cite{Dasgupta10-Play}

\subsection{Vocabulary Practice Exercises}
The number of systems that can provide vocabulary exercises to the learners is very large with several very popular commercial systems such as Babbel, DuoLingo, RosettaStone, Memrise, etc. Babbel offers 2000 to 3000 words per language, wich is comparable to most Duolingo courses. Rosetta Stone claims that you can reach up to C1 with its advanced course while the system we present here can be used by any learner, no matter how advanced. The main limitation of all these afore-mentioned systems is that the example sentences that one practices with are pre-defined. We aim for personalization. One of the advantages on the other hand, is that predefined example systems is that they can also teach grammar, which is something we do not investigate at the moment.

Spaced repetition systems like Anki are good at drilling vocabulary but they do not take the context of a word into account; even though learning a word in context is more effective \cite{nagy95-context}.

\subsection{Personalization}

The idea of a Personal eLearning Environment has been proposed by Attwell in 2007 \cite{Atwell07-personal} who assumes that it will take place in different contexts and situations and will not be provided by a single learning provider.

In web design Reinecke et al. propose culturally adaptive interfaces which are able to adapt their look and feel to suit visual preferences of a given population \cite{Reinecke13-CulturalAdaptation}. 

In mathematical education, Polozov et al. propose a technique for automatic generation of personalized word problems\cite{Polozov15-AdaptableMath}.









