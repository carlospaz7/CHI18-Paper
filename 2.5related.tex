%!TEX root=paper.tex
% \newpage
\section{Related Work}

The domain of computer assisted language learning has a rich history of applied research that aims to improve the effectiveness and efficiency of language learning through helping both teachers and students \cite{levy2013call}. In this discussion we focus on the aspects that differentiate our work from prior art.


\subsection{The Web as A Source of Content}
% Web Mining

Multiple authors have observed before that the World Wide Web represents an enormous language database at the fingertips of the students \cite{Fried08-Learner,Hira07-WebCorpora,Streit05-Browsers,Wible01-Exposure,Trus11web}. 
% Thus, the idea of augmenting texts with translations has been proposed before in various forms. 

% \begin{itemize}

	Wible et al. introduce SRP -- a tool that provides teachers and students with search capabilities for supplementary readings online \cite{Wible01-Exposure}.
	The tool discovers similar texts to one given by the teacher to provide supplementary readings that offer repeated exposure to new vocabulary.
	% The aim of SRP is to take a target vocabulary item as input and provide as output a set of texts from the corpus that contain tokens of the target vocabulary which resemble the original semantically and of course match it in part of speech.
	Unlike ours, the system is presented without any user evaluation.

	
	Streiter et al. \cite{Streit05-Browsers} argued for a a system that would support browsing the Internet and a local document repository by dynamically annotating HTML and PDF documents with open dictionaries resources. A word is annotated with translations and pictures. It argues for the creation of personal word lists and exercises. The idea is not evaluated with users. 

	
	Trusty and Truong augmented the web in a learners native language with translations of a fixed set of words in the language that they are learning \cite{Trus11web}. They show that in a two month deployment, 18 participants, learned in average 50 new words.

% \end{itemize}

Besides these research efforts, many users use browser extensions to help them understand foreign texts. D\'iaz \cite{Diaz15-Augmented} did a study of how the users augment web browsers with extensions in order to ``personalize on demand'' their browsing experience. Based on millions of web users they saw that Google Translate was the 16th most used browser extension. 
% The difference between a browser extension and the system we propose is that the system we propose is open to other applications interacting with it, while most of the extensions are an end in themselves. 
Translate allows exporting words and translations but unlike our system, does not provide an API that would allow other applications to build on top of a learner's history. Moreover, this learner history does not capture the context in which a word was translated, context which makes learning more effective \cite{nagy95-context}. 

Kindle provides translations for individual words in a text. Just like Google Translate, these translations are not available to other applications (not even devices!), and they can not be made available to the teacher of a class, or to researchers. 

There are also plenty of websites that provide texts for beginners together with translations (e.g. Veintemundos, DeutscheWelle, etc.). However, all these websites expect the human editors to decide which words to annotated with translations. This is not needed in the case of our system. 


\subsection{Interacting With Foreign Language Texts}

Augmenting foreign texts with annotations in the form of pop-ups and overlays has been found to benefit several aspects of language learning \cite{DeRidder02-Links} and reading comprehension \cite{Sanko06-Effects}.


% \begin{itemize}

	In one of the earliest such works, Nerbonne proposed Glosser -- a system which would provide dictionary information about a given word including translation, part of speech, declinations, etc. \cite{Nerb99-Assistant}.
	In a study with 22 people they observed learners using the system for twenty minutes \cite{Dokter98-UserStudy}. 
	In their work, they focus on individual words and a limited number of predefined and pre-processed texts. 
	In our work we observed a larger number of learners for a longer period of time.
	% In their study the words were previously annotated; our tools allow learners to translate sequences of words and not only individual words. 


	Azab et al.proposed a system entitled SmartReader which provides interactive annotations of English words for the advanced foreign students who learn English  \cite{Azab13-nlp}.
	Pop-ups are displayed above the selected word with information about it. 
	The study introduces and describes the system, however it does not report anything about the way the system is used.


	DeRidder \cite{DeRidder02-Links} studied the behavior of students reading with hyperlinks. The results indicate that when reading a text with highlighted hyperlinks, readers are significantly more willing to consult the gloss. Sanko \cite{Sanko06-Effects} showed that hypertextual input enhancement favourably affects vocabulary learning.
% 
	In our case, the interactive reader component we developed, considers every word to be practically a hyperlink since every word in the text can be interacted with. 
	% Our reliance on multiple third party contextual translators allows us to provide a quite 


	% Yanguas showed that multimedia gloss groups are useful in retention: they noticed that users using multimedia glosses retained and recognized significantly more of the target words than the control group \cite{Yang09-Glosses}. In our system, we do not show multimedia content, but we provide the learner with pronounciations in the cases they request them. Moreover, we present results from a one-month longitudinal study of a system, while Yanguas reported on an in-the-lab controlled experiment.



	% Horv\'ath conducted a small supervised experiment to evaluate effect of text
	% augmentation of reading speed. The results show that augmented webpage slows
	% reading speed down on average by approximately 7\%. \cite{Horva13-Enriching} 



% \end{itemize}



\subsection{Vocabulary Practice}

% Also studies show that learning a word in context is more effective \cite{nagy95-context}.

Dasgupta argues that in the context of interactive books, self-contained exercises to be included \cite{Dasgupta10-Play}. However, most of the vocabulary practice systems are disconnected from the readings of the learners. Most popular (usually commercial) systems such as Babbel, DuoLingo, RosettaStone, and Memrise are mainly focused on vocabulary drilling for beginners. 

These systems employ various types of personalized scheduling for the vocabulary exercises but when it comes to the content, they either have predefined material or they require the learner to upload the vocabulary for study (e.g. Anki, Memrise).\footnote{The systems that have predefined content usually have a limited number of words: Babbel and DuoLingo offer 2000 to 3000 words per language} 
The solution we propose adopts both personalized scheduling for the exercises and the automatic personalization of the content that is the result of retrieving the content from the readings of the learner.





% . 
% Rosetta Stone claims that one can reach past B2 with their advanced course. 
% Almost all these previous systems are targeted at the beginners and intermediates.
% The system we present here can be used by any learner, no matter how advanced. 


% The main limitation of all these afore-mentioned systems is that the example sentences that one practices with are pre-defined. 
% We aim for personalization.
% One of the advantages on the other hand, is that predefined example systems is that they can also teach grammar, which is something we do not investigate at the moment.


One promissing vein of research in vocabulary practice has recently focused on discoveirng innovative opportunities for study in order to support the busy learners. In particular, micro-learning has been used in very creative ways:
% 
	Dearman and Truong introduce a 'live wallpaper' interface always visible when a user uses the phone \cite{Dear12-ImplicitAcquisition};
% 
	Cai introduces WaitChatter providing vocabulary exercises while the user awaits instant messaging responses \cite{Cai15-wait}.
% 
The relationship between these micro-learning systems and our work is complementary: micro-learning exercises can be generated based on past readings of the learner as we showed with a smartwatch \cite{Nien16-time}.




% \subsection{Personalization}

% The idea of a Personal eLearning Environment has been proposed by Attwell in 2007 \cite{Atwell07-personal} who assumes that it will take place in different contexts and situations and will not be provided by a single learning provider.

% In web design Reinecke et al. propose culturally adaptive interfaces which are able to adapt their look and feel to suit visual preferences of a given population \cite{Reinecke13-CulturalAdaptation}. 

% In mathematical education, Polozov et al. propose a technique for automatic generation of personalized word problems\cite{Polozov15-AdaptableMath}.









