
when talking about the preferences of the students, 
look at their answers: do they know what they want? 
- i know that one of the students, said for example
that he liked sports. how many sports articles did
he read? 

- can we 


other research questions: 

- how do people use the difficulty rating? is it useful
the way it is, or we will need better strategies? 

- how often do people use the chaining translations feature? 

- how often do they use the alternate translatios feature? 

- is this a feature that is useful for the learners? 
how many articles are read? 

- quality of translations: how well do the students translate the things? the first translation might not be enough... how often do they contribute
their own translation? 

- are there articles where students just give up because 
they are too difficult? what is the density of such articles? 
can we answer this? or can we propose it as a future research 
question? 

- average translations per 100 characters? 

- manual investigation of the articles that the students have read.

- the top 5 most opened / read articles? 






