%!TEX root=paper.tex

\newpage
\section{The Testing Environment}

The population that we tested our infrastructure with consists of \stcnt  students from a public highschool in Groningen, The Netherlands. Their native language is Dutch but all speak very good English. They represent three classes of students that have the same language teacher.

At the begining of June 2017, we visited the school, and during one hour we introduced the way the tools and their usage in each of the three participating classes.

The system was used officially in class from the second week of June until the end of the month. With one or two exceptions all the students created an account and started using the system the latest on June 9th. 

The teacher asked the students to use the infrastructure as much as they liked while preparing at home and to write reports on their activity. For evey half an hour of usage, the students would have to write a brief report that they would submit to the teacher about how they spent their time. The teacher could then decide to selectively test them on the basis of their self-reported activity cards.

We deployed the system with the translations from French to English instead of Dutch since, based on our observations, translations are of higher quality between those two languages and because the students and their teacher were confortable with the idea. We made it clear to the students that if for some of them this is not convenient, they can ask us, and we will modify their personal account in such a way as to receive translations in Dutch. However, none of the students requested this.

We also asked the students to feel invited to send us feedback at any time during the deployment of the system if they encounter problems, or alternatively have ideas for improvement. Several of them emailed us. Towards the end of the month, we also deployed several focused questions using a customer opinion elicitation service called HotJar. 

\subsection{The Students}

Before creating accounts on our platform, the participants were directed to a survey form which asked them to provide personal information about their current level of knowledge, learning strategies, and interests. 

The participants in our study are 54 female and 15 male with ages below 18 representing three different classes. Based on their own self characterization, 53 students are level B1 (i.e. can understand the main points of clear standard speech, can narrate an event, an experience or a dream) and 16 are level A2 (i.e. using simple words, can describe his or her surroundings and communicate immediate needs). 


When asked whether they have favorite topics they would like to read about, half of the students mentioned such topics while the other half did not answer the question. From the topics that they mentioned as possible interests some of the more popular were: sports, music, travel, lifestyle, fashion, movies, and somebody mentioned as interest {\em no politics}.

We seeded the system with a variety of news french sources and blogs that cover these aspects: 1Jour1Actu, L'Equipe, La Blogoteque, Le Figaro, Le Monde. 











% To talk to Nienke about the other types of analysis we can do on this data